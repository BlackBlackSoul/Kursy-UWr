\documentclass[a4paper,12pt]{article}
\usepackage{polski}
\usepackage[utf8]{inputenc}
\usepackage[left = 3cm, right = 3cm, top = 2cm, bottom = 2cm]{geometry}
\usepackage{enumerate}
\usepackage{amssymb}		% pakiet do symboli
\usepackage{amsmath}		% pakiet do matmy
\usepackage{enumitem}		% punktowanie (a), (b), ...
\usepackage{nopageno}		% brak numerow stron
\usepackage{graphicx}		% wstawianie obrazkow
\usepackage{float}			% wstawianie obrazkow w dowolnym miejscu
\usepackage{titling}
%\usepackage[]{algorithm2e} 	% algorytmy :))
\usepackage{algpseudocode}	
%\usepackage{program}
%\usepackage{algorithmicx}
\usepackage{algorithm}

% nowe komendy dla wygodniejszego pisania :)
\newcommand{\subtitle}[1]{ \posttitle{ \par\end{center} \begin{center}\large#1\end{center} \vskip0.5em}}
\newcommand{\floor}[1]{\left\lfloor #1 \right\rfloor}
\newcommand{\ceil}[1]{\left\lceil #1 \right\rceil}
\newcommand{\code}[1]{\fontfamily{qcr}\selectfont\textbf{#1}\fontfamily{cmr}\selectfont}

\begin{document}
\noindent \textbf{Lista 5, zadanie 4 - Tomasz Woszczyński}\newline

\noindent \newline \textbf{Treść:} Udowodnij, że $2n-1$ porównań trzeba wykonać, aby scalić dwa ciągi $n$ elementowe w modelu drzew decyzyjnych. Zastosuj grę z adwersarzem, w której adwersarz na początku ogranicza przestrzeń danych tak, by zawierała $2n$ zestawów danych takich i by każde porównanie wykonane przez algorytm eliminowało co najwyżej jeden zestaw. \newline

\noindent \textbf{Rozwiązanie:} Mamy dwa $n$ elementowe ciągi, nazwijmy je $X$ oraz $Y$, chcemy z nich utworzyć posortowany ciąg $S$. Wiemy, że $\vert S \vert = 2n$. Gdybyśmy nie zwracali uwagi na to, że wejściowe ciągi są posortowane, to mielibyśmy aż $\binom{2n}{n}$ możliwych ułożeń wszystkich elementów, jednak taka przestrzeń danych jest zbyt duża i adwersarz będzie chciał się pozbyć jak największej ilości elementów, które może łatwo wykluczyć, aby gracz musiał wykonać jak najwięcej porównań, by znaleźć odpowiedź. Mając informację o tym, iż ciągi $X, Y$ są posortowane, możemy stworzyć własną przestrzeń zdarzeń $S_i$, tworząc każdy kolejny ciąg poprzez zamianę elementów $s_k$ oraz $s_{k+1}$ (dla $k>0$). Pierwszym, bazowym ciągiem $S_0$ będzie ciąg: 
$$x_1, y_1, x_2, y_2, \dots, x_n, y_n \equiv s_1, s_2, \ldots s_{2n}$$
\noindent Wspomnianych zamian w $2n$ elementowym ciągu możemy wykonać $2n-1$, a więc tych ciągów będzie $2n$. Kolejne ciągi $S_i$ będą wyglądać następująco:

$$S_0 = x_1, y_1, x_2, y_2, x_3, y_3,\dots, x_n, y_n$$
$$S_1 = y_1, x_1, x_2, y_2, x_3, y_3,\dots, x_n, y_n$$
$$S_2 = x_1, x_2, y_1, y_2, x_3, y_3,\dots, x_n, y_n$$
$$S_3 = x_1, y_1, y_2, x_2, x_3, y_3,\dots, x_n, y_n$$
$$S_4 = x_1, y_1, x_2, x_3, y_2, y_3,\dots, x_n, y_n$$
$$S_5 = x_1, y_1, x_2, y_2, y_3, x_3,\dots, x_n, y_n$$
$$\vdots$$
$$S_{2n-1} =  x_1, y_1, x_2, y_2, y_3, x_3,\dots, y_n, x_n$$

\noindent Mając tak przygotowane ciągi, możemy rozpatrzeć cztery możliwe odpowiedzi na pytanie "jak ułożone powinny być wartość $x_i$ oraz wartość $y_j$" (pomijamy pytania o ułożenie elementów o indeksach $i, j$ w ramach jednego ciągu, gdyż są one posortowane i nie mają w tym przypadku żadnego sensu):
\begin{enumerate}
\item $i = j$: w takiej sytuacji otrzymujemy odpowiedź, że $x_i < y_j$, dzięki czemu eliminujemy przypadek $x_i > y_j$, a więc zamianę elementów $s_{2i-1}=x_i$ oraz $s_{2i}=y_i$ w $S_0$, a więc w ciągu $S_{2i-1}$.
\item $i = j+1$: w takiej sytuacji otrzymujemy odpowiedź, że $x_i > y_j$, dzięki czemu eliminujemy przypadek, w którym $x_i < y_{i-1}$, a więc zamianę elementów  $s_{2i-2}=y_{i-1}$ oraz $s_{2i-1}=x_i$ w $S_0$, czyli ciąg $S_{2i-2}$.
\item $i > j + 1$: w takiej sytuacji otrzymujemy odpowiedź, że $x_i > y_i$ - tutaj nie eliminujemy żadnego ze swoich zestawów, gdyż kolejne zestawy to zamiana $x_i$ z $y_i$ oraz $x_{i+1}$ z $y_i$, a elementy $x_i$ oraz $y_j$ różnią się o co najmniej $2$.
\item $j > i$: w takiej sytuacji otrzymujemy odpowiedź, że $x_i < y_i$, jest to sytuacja analogiczna do przypadku trzeciego, nie eliminujemy więc żadnego zestawu, ponieważ w kolejnych elementach $s_k$ nie zamieniamy ze sobą elementów $x_i$ i $y_j$ dla $i < j$, a w $S_0$ dla $i < j$ zachodzi $x_i < y_j$.
\end{enumerate}
Dzięki rozpatrzeniu powyższych możliwości wiemy, że dla dowolnego z zapytań usuniemy maksymalnie jeden zestaw. Wiedząc, że adwersarz chce utrudnić nam życie i sprowokować do zadania jak największej ilości pytań. W tym przypadku przestrzeń wszystkich ciągów miała moc $2n$ i każde pytanie eliminowało co najwyżej jeden zestaw, co oznacza, że należy zadać co najmniej $2n-1$ pytań (lub więcej, gdy są one "głupie", a więc dotyczą informacji, które już mamy), by znaleźć odpowiedź. Skoro nie możemy eliminować więcej niż jednego zestawu na pytanie, nie możemy zadać mniejszej ilości pytań, oznacza to, że należy wykonać $2n-1$ porównań, aby w modelu drzew decyzyjnych scalić dwa posortowane ciągi $X, Y$. $\blacksquare$

\end{document}